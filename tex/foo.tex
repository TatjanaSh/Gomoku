\documentclass[11pt]{article}

\usepackage{amsmath,amssymb, a4, verbatim}
\usepackage[german]{babel}
%\usepackage[latin1]{inputenc}
\usepackage[utf8]{inputenc} % üöäß
\usepackage{listings} % für inline codelistings
\lstset{%
	basicstyle=\ttfamily,  % the size of the fonts 
	columns=fixed, % anything else is horrifying
	showspaces=false,       % show spaces using underscores?
	showstringspaces=false, % underline spaces within strings?
	showtabs=false,% show tabs within strings?
	xleftmargin=1.5em,      % left margin space
}
\lstdefinestyle{inline}{basicstyle=\ttfamily}
\newcommand{\listline}[1]{\lstinline[style=inline]!#1!}


\usepackage{caption}
\newcommand{\tinycaption}[1]{\captionsetup{labelformat=empty}\caption{#1}}

%\usepackage{color}
%\usepackage{epsfig} % eps
\usepackage{graphicx} % eps
%\usepackage[shortcuts]{extdash}
%\usepackage{dsfont}
%\usepackage{epstopdf} % eps
%\usepackage[pdf]{pstricks} % eps
%\usepackage{auto-pst-pdf}
\usepackage{mathtools}
\usepackage{dsfont} % $ \mathds{1} $
\usepackage{icomma}
\usepackage{tikz}
%\usepackage{pgfplots}
%\pgfplotsset{compat=1.8}
\usepackage[bottom]{footmisc} % put footnotes at the bottom of page
\usepackage{nicefrac} % für brüche die aussehen wie prozentzeichen
% \usepackage{ps2pdf}
\usetikzlibrary{automata,positioning}

\usepackage{algorithmicx}
\usepackage{algpseudocode}
\usepackage{algorithm}

\usepackage{multicol}
\usepackage{wrapfig} % make stuff float
\usepackage{placeins} % stop stuff from floating
\usepackage{seqsplit} % very long numbers
\usepackage{framed} % begin{framed}

%  Headings and Footings :
\usepackage{fancyhdr}
\headheight15pt
\lhead{AWP oder so, \ueberschrift}

\chead{}
\rhead{\thepage}
\renewcommand{\headrulewidth}{.4pt}

\lfoot{\today}
\cfoot{}
\rfoot{Name...}
\renewcommand{\footrulewidth}{.4pt}

%----------------------------------------------------------------

\textwidth16.5cm
\oddsidemargin0.cm
\evensidemargin0.cm

\parindent0cm

\newcommand{\R}{ {\mathbb R} }
\newcommand{\C}{ {\mathbb C} }
\newcommand{\1}{ {\mathds{1}} }
\newcommand{\abs}[1]{\lvert#1\rvert}
\newcommand{\norm}[1]{\left\lVert#1\right\rVert}
\newcommand{\xt}{\tilde{x}}
\newcommand{\dotleq}{\dot{\leq}}
\newcommand{\m}{\hphantom{-} }

\newcommand{\dashfill}[1]{\vspace{11pt}\def\dashfill{\cleaders\hbox{#1}\hfill}\hbox to \hsize{\dashfill\hfil}\vspace{11pt}}
\newcommand{\scdot}{\!\cdot\!}

\newcommand{\sig}{\text{signum}}
\newcommand{\rot}{}

% ------------------  edit Ueberschrift ---------------------
\newcommand{\ueberschrift}{Dokumentation oder so}


\usepackage{pgf-umlcd}
% -----------------------------------------------------------
\begin{document}
	\pagestyle{fancy}
	
	\begin{tikzpicture}
		\begin{package}{application}
			\begin{package}{model}
				\begin{class}[text width=\linewidth]{SpielStein}{0, 0}
					\attribute{- farbe : int}
					\attribute{- img : Image}
					\attribute{- blackStone : static Image }
					\attribute{- whiteStone : static Image }
					
					\operation{+ getColor() : final int}
					\operation{+ getImage() : final Image}
				\end{class}
			\end{package}%model
		\end{package}%application
	\end{tikzpicture}
	
	\begin{tikzpicture}
		\begin{package}{application}		
			\begin{package}{view}
				\begin{class}[text width=\linewidth]{GewinnerController}{0, 0}
					\attribute{- gewinnerPane : AnchorPane }
					\attribute{- abbrechenButton : Button }
					\attribute{- neuButton : Button }
					\attribute{- gewinneriView :ImageView }
					\attribute{- gewinnerText : Text }				
					\attribute{- spielController : SpielController }
					\attribute{- gewinnerStage : Stage}
					\operation{handleAbbrechenButton(ActionEvent event) : void}
					\operation{handleNeuButton(ActionEvent event) : void}
					\operation{setGewinnerImage(Image image) : void}		
					\operation{setDialogStage(Stage gewinnerStage) : void}
					\operation{setGewinnerText(String s) : void}
					\operation{setDialogSpielController(SpielController spielController):void}
				\end{class}
			\end{package}%view
		\end{package}%application
	\end{tikzpicture}
\end{document}














































