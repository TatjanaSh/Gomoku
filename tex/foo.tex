\documentclass[11pt]{article}

\usepackage{amsmath,amssymb, a4, verbatim}
\usepackage[german]{babel}
%\usepackage[latin1]{inputenc}
\usepackage[utf8]{inputenc} % üöäß
\usepackage{listings} % für inline codelistings
\lstset{%
	basicstyle=\ttfamily,  % the size of the fonts 
	columns=fixed, % anything else is horrifying
	showspaces=false,       % show spaces using underscores?
	showstringspaces=false, % underline spaces within strings?
	showtabs=false,% show tabs within strings?
	xleftmargin=1.5em,      % left margin space
}
\lstdefinestyle{inline}{basicstyle=\ttfamily}
\newcommand{\listline}[1]{\lstinline[style=inline]!#1!}


\usepackage{caption}
\newcommand{\tinycaption}[1]{\captionsetup{labelformat=empty}\caption{#1}}

%\usepackage{color}
%\usepackage{epsfig} % eps
\usepackage{graphicx} % eps
%\usepackage[shortcuts]{extdash}
%\usepackage{dsfont}
%\usepackage{epstopdf} % eps
%\usepackage[pdf]{pstricks} % eps
%\usepackage{auto-pst-pdf}
\usepackage{mathtools}
\usepackage{dsfont} % $ \mathds{1} $
\usepackage{icomma}
\usepackage{tikz}
%\usepackage{pgfplots}
%\pgfplotsset{compat=1.8}
\usepackage[bottom]{footmisc} % put footnotes at the bottom of page
\usepackage{nicefrac} % für brüche die aussehen wie prozentzeichen
% \usepackage{ps2pdf}
\usetikzlibrary{automata,positioning}

\usepackage{algorithmicx}
\usepackage{algpseudocode}
\usepackage{algorithm}

\usepackage{multicol}
\usepackage{wrapfig} % make stuff float
\usepackage{placeins} % stop stuff from floating
\usepackage{seqsplit} % very long numbers
\usepackage{framed} % begin{framed}

%  Headings and Footings :
\usepackage{fancyhdr}
\headheight15pt
\lhead{AWP oder so, \ueberschrift} % TODO

\chead{}
\rhead{\thepage}
\renewcommand{\headrulewidth}{.4pt}

\lfoot{\today}
\cfoot{}
\rfoot{Name...} %TODO
\renewcommand{\footrulewidth}{.4pt}

%----------------------------------------------------------------

\textwidth16.5cm
\oddsidemargin0.cm
\evensidemargin0.cm

\parindent0cm

\newcommand{\R}{{\mathbb R}}
\newcommand{\C}{{\mathbb C}}
\newcommand{\1}{{\mathds{1}}}
\newcommand{\abs}[1]{\lvert#1\rvert}
\newcommand{\norm}[1]{\left\lVert#1\right\rVert}
\newcommand{\xt}{\tilde{x}}
\newcommand{\dotleq}{\dot{\leq}}
\newcommand{\m}{\hphantom{-}}

\newcommand{\dashfill}[1]{\vspace{11pt}\def\dashfill{\cleaders\hbox{#1}\hfill}\hbox to \hsize{\dashfill\hfil}\vspace{11pt}}
\newcommand{\scdot}{\!\cdot\!}

\newcommand{\sig}{\text{signum}}
\newcommand{\rot}{}

% ------------------  edit Ueberschrift ---------------------
\newcommand{\ueberschrift}{Dokumentation oder so} % TODO


\usepackage{varwidth}
\newcommand{\umrandet}[1]{\pbox{\textwidth}{\boxed{\text{#1}}}}

\newcommand{\aGreatFramedBox}[1]{\fbox{\begin{varwidth}{\textwidth}#1\end{varwidth}}\ignorespacesafterend}
\newcommand{\umramt}[1]{\begin{minipage}{\widthof{\aGreatFramedBox{#1}}}\aGreatFramedBox{#1}\ignorespacesafterend\end{minipage}\ignorespacesafterend}

\usepackage{pgf-umlcd}
% variables to tweak appearance of uml stuff
\newcommand{\umlScale}{0.9}
\newcommand{\umlScaleSpielController}{0.6}
\newcommand{\umlWidth}{.74\linewidth}
% -----------------------------------------------------------
\begin{document}
	\pagestyle{fancy}
	
	\begin{tikzpicture}[thick, scale=\umlScale, every node/.style={scale=\umlScale}]
		%\begin{package}{application}
			%\begin{package}{model}{0,-10}
				\begin{class}[text width=\umlWidth]{SpielStein}{0, 0}
					\attribute{$-$ farbe : int}
					\attribute{$-$ img : Image}
					\attribute{$-$ blackStone : static Image}
					\attribute{$-$ whiteStone : static Image}
					
					\operation{+ getColor() : final int}
					\operation{+ getImage() : final Image}
				\end{class}
			%\end{package}%model
		%\end{package}%application
	\end{tikzpicture}
	
	\begin{tikzpicture}[thick, scale=\umlScale, every node/.style={scale=\umlScale}]
		%\begin{package}{application}
			%\begin{package}{view}
				\begin{class}[text width=\umlWidth]{GewinnerController}{0, 0}
					\attribute{$-$ gewinnerPane : AnchorPane}
					\attribute{$-$ abbrechenButton : Button}
					\attribute{$-$ neuButton : Button}
					\attribute{$-$ gewinneriView :ImageView}
					\attribute{$-$ gewinnerText : Text}				
					\attribute{$-$ spielController : SpielController}
					\attribute{$-$ gewinnerStage : Stage}
					\operation{+ handleAbbrechenButton(event : ActionEvent) : void}
					\operation{+ handleNeuButton(event : ActionEvent) : void}
					\operation{+ setGewinnerImage(image Image) : void}		
					\operation{+ setDialogStage(gewinnerStage : Stage) : void}
					\operation{+ setGewinnerText(s : String) : void}
					\operation{+ setDialogSpielController(spielController : SpielController):void}
				\end{class}
			%\end{package}%view
		%\end{package}%application
	\end{tikzpicture}
	
	\begin{tikzpicture}[thick, scale=\umlScale, every node/.style={scale=\umlScale}]
		%\begin{package}{application}	
			%\begin{package}{model}
				\begin{class}[text width=\umlWidth]{Brett}{0, 0}
					\attribute{$-$ \_dim : int}
					\attribute{$-$ \_spieler : int}
					\attribute{$-$ \_brett : SpielStein[ ][ ]}
					\attribute{$-$ \_gitterVert : List$<$Line$>$}
					\attribute{$-$ \_gitterHorz : List$<$Line$>$}
					\attribute{$-$ \_gitter : List$<$Line$>$}
					\attribute{$-$ \_SpielZuege : List$<$SpielZug$>$} 
					\attribute{$-$ \_gitterWeite : double}
					\attribute{$-$ \_randX : double}
					\attribute{$-$ \_randY : double} 
					\attribute{$-$ \_CheckAdjacent : boolean}
					\operation{+ Brett(dim : int, x : double, y : double)} 
					\operation{+ redrawGitter(x : double, y : double) : void}
					\operation{+ roundCoord(x : double, y : double) : double[ ]}
					\operation{+ steinAt(int x, int y) : SpielStein}
					\operation{+ steinAt(double x, double y) : SpielStein}
					\operation{$-$ steinSet(int x, int y, SpielStein s) : boolean}
					\operation{+ makeMove(SpielZug zug) : boolean}
					
					\operation{+ printMoves() : void}
					\operation{+ getNextMoveColour() : int}
					\operation{+ List$<$SpielZug$>$ getSpielZuege() : final}
					\operation{+ getDim() : int}
					\operation{+ getGitter() : List$<$Line$>$}
					\operation{+ getGitterWeite() : double}
					\operation{+ getRandX() : double}
					\operation{+ getRandY() : double}
					\operation{+ getSpieler() : final int}
					\operation{+ getBrett() : final SpielStein[ ][ ]}
				\end{class}
				\begin{class}[text width=\umlWidth]{+ static SpielZug}{0, -14.5}
					\attribute{+ x, y : int}
					\attribute{+ stein:SpielStein}
					\attribute{+ iView:ImageView}
					\operation{+ SpielZug(x:int, y:int, stein:SpielStein, iView:ImageView)}
					\operation{+ toString():String}
				\end{class}
			%\end{package}%model
		%\end{package}%application
	\end{tikzpicture}

	\begin{tikzpicture}[thick, scale=\umlScale, every node/.style={scale=\umlScale}]
		%\begin{package}{application}	
			%\begin{package}{model}
				\begin{class}[text width=\umlWidth]{Options}{0, 0}
					\attribute{$-$ \_menge : HashSet$<$Tupel$>$}
					\operation{+ Options()}
					\operation{+ setOption(name : String , objekt : Object) : void}
					\operation{+ getOption(name : String) : Object}
					\operation{+ printOption(name : String) : void}
					\operation{+ toString() : String}
				\end{class}
				\begin{class}[text width=\umlWidth]{-Tupel}{0, -4.0}
					\attribute{+ name : String}
					\attribute{+ objekt : Object}
					\operation{+ Tupel(name : String, objekt : Object)}
					\operation{+ hashCode() : int}
					\operation{+ equals(Object obj) : boolean}
					\operation{$-$ getOuterType() : Options}
					\operation{+ toString() : String}
				\end{class}
			%\end{package}%model
		%\end{package}%application
	\end{tikzpicture}
	
	\begin{tikzpicture}[thick, scale=\umlScale, every node/.style={scale=\umlScale}]
		%\begin{package}{application}
			\begin{class}[text width=\umlWidth]{Main}{0, 0}
				\attribute{+ optionen : static Options}
				\attribute{+ primaryStage : static Stage}
				\operation{+ start(primaryStage : Stage) : void}
				\operation{+ main(args : String[]) : static void}
			\end{class}
		%\end{package}%application
	\end{tikzpicture}
	
	\begin{tikzpicture}[thick, scale=\umlScale, every node/.style={scale=\umlScale}]
		%\begin{package}{application}
			%\begin{package}{model}
				\begin{class}[text width=\umlWidth]{SpielAI}{0, 0}
					\attribute{$-$ \_brett : Brett}
					\attribute{$-$ \_possibleMoves : ArrayList$<$LinkedHashSet$<$Savegame$>>$}
					\operation{+ SpielAI(brett : Brett) }
					\operation{+ generateNextMoves() : void}
					\operation{+ updateMoves() : void}
					\operation{+ getBestMoves() : Integer[ ][ ]}		
					\operation{+ addDoubleArray(a : Double[ ][ ], b : Double[ ][ ]) : static Double[ ][ ] }
					\operation{+ multDoubleArray(a : Double[ ][ ], f : double) : static void }
					\operation{+ twoDeepCloneDouble(a : Double[ ][ ]) : static Double[ ][ ] }
					\operation{+ printDoubleArray(a : Double[ ][ ]) : static void }
				\end{class}
				\begin{class}[text width=\umlWidth]{Savegame}{0, -6}
					\attribute{$-$ moveNr : int}
					\attribute{$-$ steine : int[ ][ ]}
					\attribute{$-$ spielerAnz : int}
					\attribute{$-$ dim : int}
					\attribute{$-$ nextMove : int[ ]}
					\operation{+ Savegame(brett : Brett)}
					\operation{+ generateNextMoves() : LinkedHashSet$<$Savegame$>$}
					\operation{+ generateHeuristic() : Double[ ][ ]}
					\operation{+ hashCode() : int}
					\operation{+ toString() : String}
					\operation{+ equals(obj : Object) : boolean}
				\end{class}
			%\end{package}%model
		%\end{package}%application
	\end{tikzpicture}
	
	\begin{tikzpicture}[thick, scale=\umlScaleSpielController, every node/.style={scale=\umlScaleSpielController}]
		%\begin{package}{application}
			%\begin{package}{view}
				\begin{class}[text width=\umlWidth]{SpielController}{0, 0}
					\attribute{$-$ mitteBeginnCheckBox : CheckBox }
					\attribute{$-$ backgroundImage : ImageView }
					\attribute{$-$ tabPaneSwitch : TabPane }
					\attribute{$-$ ueberTab : Tab }
					\attribute{$-$ anlegenCheckBox : CheckBox }
					\attribute{$-$ brettGroesseLabel : Label }
					\attribute{$-$ einstellungenAnchorPane : AnchorPane }
					\attribute{$-$ gameAnchorPane : AnchorPane }
					\attribute{$-$ aiCheckBox : CheckBox }
					\attribute{$-$ helpTab : Tab }
					\attribute{$-$ gameTab : Tab }
					\attribute{$-$ brettGroesseTextField : TextField }
					\attribute{$-$ zweiSpielerButton : RadioButton }
					\attribute{$-$ stoneImage : ImageView }
					\attribute{$-$ brettGroesseBox : ComboBox$<$String$>$ }
					\attribute{$-$ bild2Button : ToggleButton }
					\attribute{$-$ einstellungenTab : Tab }
					\attribute{$-$ anzahlReiheTextField : TextField }
					\attribute{$-$ bild1Button : ToggleButton }
					\attribute{$-$ einSpielerButton : RadioButton }
					\attribute{$-$ aiButton : RadioButton }
					\attribute{$-$ hilfeText : TextArea }
					\attribute{$-$ uberText : TextArea }
					\attribute{$-$ neuButton : Button }
					\attribute{$-$ spielStartenButton : Button }
					\attribute{$-$ startButton : Button }
					\attribute{$-$ newGameButton : Button }
					\attribute{$-$ pauseGameButton : ToggleButton }
					\attribute{$-$ zuruecksetzenButton : Button }
					\attribute{$-$ wrapAnchorPane : AnchorPane }
					\attribute{$-$ aiSpeedSlider : Slider }
					\attribute{$-$ radioButtonGroup : final ToggleGroup }
					\attribute{$-$ bildGroup : final ToggleGroup }
					\attribute{$-$ choiceBoxOptions : ObservableList$<$String$>$ } 
					\attribute{$-$ spielbrett : Brett }
					\attribute{$-$ gameDone : boolean }
					\attribute{$-$ s : SpielStein }
					\attribute{$-$ currWidth, currHeight : double }
					\attribute{$-$ lastPlayed : ImageView }
					\attribute{$-$ winningStone : List$<$ImageView$>$ }
					\attribute{$-$ gegner : SpielAI }
					\attribute{$-$ lastTime : static long }
					\attribute{$-$ aiPaused : boolean }
					\attribute{$-$ zweiAiTimer : AnimationTimer }
					 
					\operation{ $-$ initialize() : void }
					\operation{ $-$ standardEinstellungen() : void }
					\operation{ + bildeBrett() : void }
					\operation{ $-$ handleSpielerAnzahlButton(ActionEvent event) : void }
					\operation{ $-$ handleBrettGroesseBox(ActionEvent event) : void }
					\operation{ $-$ handleBrettGroesseFeld(ActionEvent event) : void }
					\operation{ $-$ handleSpielregeln(ActionEvent event) : void }
					\operation{ $-$ handleBackground(ActionEvent event) : void }
					\operation{ + neustart() : void }
					\operation{ $-$ handleSpielStartenButton() : void }
					\operation{ $-$ handleZuruecksetzenButton(ActionEvent event) : void }
					\operation{ $-$ handleStartButton() : void }
					\operation{ $-$ disable() : void }
					\operation{ $-$ enable() : void }
					\operation{ $-$ handleNewGameButton() : void }
					\operation{ $-$ handlePauseGameButton(ActionEvent event) : void }
					\operation{ $-$ handleMouseMoved(MouseEvent event) : void }
					\operation{ $-$ handleSizeChanged() : void }
					\operation{ $-$ handleSizeChanged(boolean forceIt) : void }
					\operation{ $-$ updatePlayMarkers() : void }
					\operation{ $-$ handleDragDetected(MouseEvent event) : void }
					\operation{ $-$ handleMouseClicked(int x, int y) : void }
					\operation{ $-$ handleMouseClicked(MouseEvent event) : void }
					\operation{ $-$ letAImakeMove() : void }
					\operation{ $-$ handleGewinner() : boolean }
					\operation{ $-$ handleGewinner(boolean unentschieden) : boolean }
					\operation{ $-$ checkIfGewinner() : boolean }
					\operation{ $-$ handleKeyPressed(KeyEvent event) : void }
					\operation{ $-$ handleKeyReleased(KeyEvent event) : void }
				\end{class}
			%\end{package}%model
		%\end{package}%application
	\end{tikzpicture}
	

		

	
	\begin{wrapfigure}{L}{\textwidth}
		getBestMoves():\\
		\umramt{\begin{varwidth}{\textwidth}
			\begin{algorithmic}[1]
				\State $h\leftarrow$ generateHeuristic()
				\State $max \leftarrow-\infty$
				\State $n\leftarrow 0$
				\For{$i \in \{0, dim(h)\} $}
					\For{$j \in \{0, dim(h)\} $}
						\If{$max < h_{i, j}$}
							\State $max \leftarrow h_{i, j}$
							\State $n\leftarrow 1$
						\ElsIf{$max=h_{i, j}$}
							\State $n++$
						\EndIf
					\EndFor
				\EndFor
				\State $erg :\in {\mathbb{N}}^{\text{n}\times 2}$
				\For{$i \in \{0, dim(h)\} $}
					\For{$j \in \{0, dim(h)\} $}
						\If{ $max = h_{i, j}$ }
							\State 	$n \leftarrow n-1$
							\State 	$erg_{n, 0}\leftarrow i$
							\State 	$erg_{n, 1}\leftarrow j$
						\EndIf
					\EndFor
				\EndFor
				\State\Return erg
			\end{algorithmic}
		\end{varwidth}}%
	\end{wrapfigure}
	
\end{document}














































